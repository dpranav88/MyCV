%%%%%%%%%%%%%%%%%%%%%%%%%%%%%%%%%%%%%%%%%
% "ModernCV" CV and Cover Letter
% LaTeX Template
% Version 1.11 (19/6/14)
%
% This template has been downloaded from:
% http://www.LaTeXTemplates.com
%
% Original author:
% Xavier Danaux (xdanaux@gmail.com)
%
% License:
% CC BY-NC-SA 3.0 (http://creativecommons.org/licenses/by-nc-sa/3.0/)
%
% Important note:
% This template requires the moderncv.cls and .sty files to be in the same 
% directory as this .tex file. These files provide the resume style and themes 
% used for structuring the document.
%
%%%%%%%%%%%%%%%%%%%%%%%%%%%%%%%%%%%%%%%%%

%----------------------------------------------------------------------------------------
%	PACKAGES AND OTHER DOCUMENT CONFIGURATIONS
%----------------------------------------------------------------------------------------
\documentclass[11pt,a4paper,sans]{moderncv} % Font sizes: 10, 11, or 12; paper sizes: a4paper, letterpaper, a5paper, legalpaper, executivepaper or landscape; font families: sans or roman

\moderncvstyle{banking} % CV theme - options include: 'casual' (default), 'classic', 'oldstyle' and 'banking'
\moderncvcolor{blue} % CV color - options include: 'blue' (default), 'orange', 'green', 'red', 'purple', 'grey' and 'black'

\usepackage{lipsum} % Used for inserting dummy 'Lorem ipsum' text into the template

\usepackage[scale=0.75]{geometry} % Reduce document margins
%\setlength{\hintscolumnwidth}{3cm} % Uncomment to change the width of the dates column
%\setlength{\makecvtitlenamewidth}{10cm} % For the 'classic' style, uncomment to adjust the width of the space allocated to your name

%\usepackage{fontawesome}

%----------------------------------------------------------------------------------------
%	NAME AND CONTACT INFORMATION SECTION
%----------------------------------------------------------------------------------------

\firstname{\LARGE{Pranav}} % Your first name
\familyname{\LARGE{Deshpande}} % Your last name

% All information in this block is optional, comment out any lines you don't need
\title{\LARGE{Curriculum Vitae}}
\address{Krishnai-539a, Amboli Road, Ajara, Maharashtra, India 416505}{}
\mobile{(+91) 7276405699 }
\phone{(+91) 2323 244356}
%\fax{(000) 111 1113}
\email{dpranav88@yahoo.co.in}
\homepage{https://www.linkedin.com/in/dpranav88/}{LinkedIn} % The first argument is the url for the clickable link, the second argument is the url displayed in the template - this allows special characters to be displayed such as the tilde in this example
%\extrainfo{additional information}
\photo[70pt][0.4pt]{pictures/picture} % The first bracket is the picture height, the second is the thickness of the frame around the picture (0pt for no frame)
\quote{"Strength is Life, Weakness is Death!" - Swami Vivekananda}

%----------------------------------------------------------------------------------------

\begin{document}

\makecvtitle % Print the CV title

%%----------------------------------------------------------------------------------------
%%	CAREER OBJECTIVE
%%----------------------------------------------------------------------------------------
%
%\section{\textbf{Career Objective}}
%\cventry{}{}{}{}{}{An IITian looking for best opportunity in signal processing and computer vision.}
%----------------------------------------------------------------------------------------
%	EDUCATION SECTION
%----------------------------------------------------------------------------------------

\section{\textbf{Education}}

\cventry{2013--2015}{Masters of Technology}{Indian Institute of Technology}{Bhubaneswar}{\textit{CGPA -- 8.3}}{Electronics and Communication}  % Arguments not required can be left empty
\cventry{2006--2010}{Bachelor of Engineering}{University of Mumbai}{Mumbai}{\textit{Percentage -- 66.39}}{Electronics Engineering}


%----------------------------------------------------------------------------------------
%	COMPUTER SKILLS SECTION
%----------------------------------------------------------------------------------------

\section{\textbf{Skills}}
\subsection{\textbf{Domain}}
\cvitem{Signals}{Digital Signal Analysis and Processing (Motion Sensors, Basic Biomedical and Audio Signal Processing, Basic Image Processing), Sensor Fusion (IMU)}
\cvitem{Filtering}{Digital filter design (IIR, FIR), Filter Banks, Adaptive Filtering}
\cvitem{Data Science}{Feature Extraction, Basics of Machine Learning, Soft and Evolutionary Computing}
%\cvitem{Basics}{Random Signal Modeling, Estimation Theory, Convex Optimization,Linear Algebra}
\subsection{\textbf{Tools}}
\cvitem{MATLAB}{Code conversion (Coder), Simulation and Scripting with Toolboxes: Signal Processing, DSP System, Statistics and Machine Learning, etc.}
\cvitem{\textsc{Python}}{Basic of NumPy, SciPy, pandas, matplotlib, OpenCV, scikit-learn, TensorFlow (Keras)}
\cvitem{C/C++}{Mainly worked with C for Signal Processing Algorithm Development, Sensor Library Integration, Basic openCV, CMSIS DSP etc. Basic level competency in C++}
\cvitem{Hardware}{Basics of STM32F407 Discovery Board, Arduino UNO, Raspberry Pi 3 etc.}
\cvitem{Others}{Basics of Shell Scripting, GNU Make, SQLite, Jenkins, Git, \LaTeX, scrum practices, Ubuntu and Microsoft Windows.}

\newpage
%----------------------------------------------------------------------------------------
%	WORK EXPERIENCE SECTION
%----------------------------------------------------------------------------------------

\section{\textbf{Professional Experience}}

\subsection{\textbf{Industrial}}
\cventry{Mar 2022--Dec 2024}{Member of Technical Staff Software System Design Engineer}{AMD India Pvt. Ltd.}{Bengaluru}{}{Serving as an Algorithm Developer.
\center{\textsc{Key Result Area:}}
Algorithm development for AMD Sensor Fusion Hub Library
\begin{itemize}
\item Provided Algorithm solution for  
\begin{itemize}
\item Hand Gesture Detection using ToF Array Sensors for Laptop usecase.
\item Laptop state detection, viz. on table, on lap, in bag etc with only accelerometer sensor. 
\item Laptop Mode detection, viz. book, tent, tablet etc. using two accelerometers. 
\item Laptop Screen Orientation Detection using single accelerometer. 
\end{itemize}
\item \textbf{Designed signal pre-processing modules}, viz. offset removal and noise filtering for accelerometer, gyroscope and distance  sensors.
\item Designed and executed POC for Laptop device state detection, viz. on table, on lap, in bag etc..
\item Wrote software in MATLAB and Python scripts to develop simulators for Sensor Fusion Library. Also implemented C codes for same.
\item Trained Machine Learning Models for hand gesture detection using ToF Sensor for laptop mouse such as scroll, click etc.
\item Integrated third party sensor libraries on AMD's platform.
\item \textbf{Innovation}
\begin{itemize}
\item Presented a conference paper at AMD's Internal GTAC 2023 Conference.
\item Presented a poster at at AMD's Internal AATC 2023 Conference.
\item Submitted 3 innovation reports which are under scrutiny of AMD's patent team.
\item Submitted papers and research work at AMD's iExpo, AATC 2023, 2024 platforms.
\end{itemize}
\item Lead and mentored a team of two Juniors.
\item Improved software devlopment cycle process by incorporating automation (Python) such as BIOS creation, SW release etc. This reduced time consuming manual process and improved entire team's efficiency.
\end{itemize}}

\vspace{0.5cm}

\cventry{Sept 2021--Mar 2022}{Software Engineer}{Tektronix India Pvt. Ltd.}{Bengaluru}{}{Served as a Software Developer for PCIe team.}


\cventry{Sept 2017--Sept 2021}{Associate Research Engineer}{Brigosha Technologies Pvt. Ltd.}{Bengaluru}{}{Serving as an Algorithm Developer at client location, i.e. \texttt{Robert Bosch Engineering and Business Solutions Pvt. Ltd. (RBEI), Coimbatore.}
\center{\textsc{Key Result Area:}}
Algorithm development for Sensor Fusion Library
\begin{itemize}
\item Provided innovative algorithm solution for  
\begin{itemize}
\item Laptop Gesture and state detection with only accelerometer sensor. (Two Indian \textbf{patents} published.)  
\item FlipCam and FastFlash gestures detection with only accelerometer sensor. (Indian \textbf{patent} published.)  
\item 9DoF based heading angle error reduction under slowly varying magnetic field. (Indian \textbf{patent} published.)
\item Temperature compensated gyroscope bias estimation.(Submitted to RBEI IP Dept.)  
\end{itemize}
\item \textbf{Developed automated MATLAB framework to convert trained keras CNN model into c code convertible MATLAB code.}
\item \textbf{Designed signal pre-processing modules}, viz. offset removal and noise filtering for accelerometer, gyroscope and pressure sensors. Specified signal and noise characteristics for different applications, viz. cars, drones and robots.
\item Done fine tuning of \textbf{sensor fusion algorithms} such as Kalman filters, Mahony filters.
\item Wrote Python script for fitting a curve to gyroscope bias vs. temperature characteristics using \textbf{Linear Regression}.
\item \textbf{Designed Kalman Filter state model} for online estimation of temperature coefficient of gyroscope sensor offsets (TCO).  
\item Developed harsh acceleration and sudden braking detection algorithm for driving quality assessment. 
%\item Designed and executed POC for Laptop device state detection, viz. on table, on lap, in bag etc..
% \item Implemented Python and C++ software for Semi-Global Matching (SGM) based object to lens distance estimation.
\item Wrote scikit-learn based Python scripts for motion activity and event classification.
\item Wrote software in MATLAB and Python scripts to develop simulators for Motion Sensor Library. Also implemented C codes for same.
\item Trained CNN for hand  gesture detection for video conferencing application
\item Developed PoC for facial landmark based speech activity detection from video stream using MediaPipe.
\end{itemize}}
%------------------------------------------------
\cventry{Nov 2016-- Aug 2017}{Research Scientist}{\textsc{Atreya Innovations Pvt. Ltd.}}{Pune}{}{Had an experience of \textbf{10 months}. Primarily worked on the development of multi-modal signal analysis framework, i.e. Naadi (the Radial Artery Pulse), Voice and Image of the subject, for Naadi  and other Parikshas (Pulse based Diagnosis in Ayurveda). For this, we interacted with a team of Ayurvedic Doctors. The main tasks include data collection (, i.e. Pulse signals, Face and tongue images, voice) at medical camps and analyzing them by feature extraction and machine learning algorithms.
\center{\textsc{Key Result Area:}}
Multi-modal signal based Health Analysis: Machine Learning Approach
\begin{itemize}
\item Done literature survey on multi-modal signal analysis (, viz. radial artery pulse signals (\textit{Naadi}), voice samples, face and tongue images) for health (\textit{Prakruti}) diagnosis.
\item Provided technical specifications and rules for data quality for creating database with voice and image samples. 
\item Implemented MATLAB software for signal conditioning and pre-processing modules for voice and pulse signals data.
\item Implemented MATLAB software for voice activity detection algorithm.
\item Wrote Python script for extracting voice specific features, viz. MFCC, time and frequency domain features. 
\item Wrote MATLAB scripts for time domain, frequency domain and geometric feature extraction for pulse rate variability analysis.
\item Wrote Python scripts for different supervised and unsupervised machine learning algorithms with scikit-learn library for data classification using multi-modal signal features.
\item Set up regression model to establish relation between of Ayurvedic definitions and voice and pulse specific features.
%\item Developed MLP-based Neural Network model in scikiti-learn for predicting health category using features extracted from and Face images.
\item Guided two intern M.Sc. projects related with pulse parameter and rate variability with machine learning approach.
\end{itemize}}

%------------------------------------------------
\subsection{\textbf{Academic Research}}
\cventry{2013--2016}{Research Scholar}{\textsc{Indian Institute of Technology}}{Bhubaneswar}{}{Had around \textbf{03 years} of research experience while working as a M. Tech. scholar and Ph.D. scholar at IIT Bhubaneswar. Published one conference and two journals. Explored areas like stochastic signal modeling, speech and biomedical signal processing and analysis, Image processing.}

\subsection{\textbf{Teaching}}
\cventry{2010--2016}{Teaching Assistant and Lecturer}{}{}{}{Had around \textbf{05 years} teaching experience and taught subjects like signal processing, microprocessors and analog circuit design.}

%----------------------------------------------------------------------------------------
%	AWARDS SECTION
%----------------------------------------------------------------------------------------

%\section{Awards}
%
%\cvitem{2011}{School of Business Postgraduate Scholarship}
%\cvitem{2010}{Top Achiever Award -- Commerce}

%----------------------------------------------------------------------------------------
%	Publications SECTION
%----------------------------------------------------------------------------------------

\section{\textbf{Extra Certifications}}
\cvitem{1}{\textbf{TensorFlow in Practice Specialization} by Laurence Moroney, deeplearning.ai, on Coursera, \newline{} (https://www.coursera.org/account/accomplishments/specialization/certificate/DRE9N5W8K3GV)}

\cvitem{2}{\textbf{Deep Learning Specialization} by Andrew Ng, deeplearning.ai, on Coursera, \newline{} (https://www.coursera.org/account/accomplishments/specialization/certificate/K3AVXND549SD)}

\cvitem{3}{\textbf{Machine Learning} by Andrew Ng, Stanford University on Coursera,\newline{} (https://www.coursera.org/account/accomplishments/verify/JBRXBMHC86RY)}

%\cvitem{2018}{\textbf{Fundamentals of Parallelism on Intel Architecture} by Intel on Coursera, (Complete by July 2018)}

%\cvitem{2:}{\textbf{Introduction to Parallel Programming using GPGPU and CUDA} on Udemy,\newline{} (https://www.udemy.com/certificate/UC-76HISUU8/)}

%\cvitem{2018}{\textbf{Introduction to Embedded Systems Software and Development Environments} university of Coursera boulder on Coursera, (September 2018)}

%----------------------------------------------------------------------------------------
%	Publications SECTION
%----------------------------------------------------------------------------------------

\section{\textbf{Patents and Publications}}
\subsection{\textbf{Patents}}

\cvitem{1}{\textbf{INA 202241037881} , ``A POWER CONTROL APPARATUS AND A POWER CONTROL METHOD FOR A MOBILE DEVICE ", Jan 5, 2024.}
\cvitem{2}{\textbf{INA 202141027914} , ``A LAPTOP POSITION DETECTION SYSTEM ", Dec 30, 2022}
\cvitem{3}{\textbf{INA 201941053086 } , ``A SYSTEM AND METHOD FOR DETECTING GESTURES USING ACCELEROMETER", Jun 25, 2021 }
\cvitem{4}{\textbf{INA 201841040798} , ``A METHOD OF FILTERING STATIC MAGNETIC FIELD DISTORTION DURING CALCULATION OF HEADING ANGLE", May 1, 2020.}

\subsection{\textbf{International Journals}}

\cvitem{1}{\textbf{Pranav S. Deshpande} and M. S. Manikandan, ``Effective Glottal Instant Detection and Electroglottographic Parameter Extraction for Automated Voice Pathology Assessment", \textit{IEEE J. Biomed. Health Inform.,} vol.PP, no.99, pp.1-1, Jan 2017.}
\cvitem{2}{\textbf{Pranav S. Deshpande} and M. S. Manikandan, ``Glottal Opening Instants Detection from Speech Signals Using Variational Mode Decomposition", \textit{IEEE Trans. Instrum. Meas.,} (To be submitted), 2018.}
\cvitem{3}{M. S. Manikandan, B. Ramkumar, \textbf{Pranav S. Deshpande}, T. Choudhary, ``Robust Detection of Premature Ventricular Contractions Using Sparse Signal Decomposition and Temporal Features", \textit{IET Healthcare Technology Letters,} vol.2, no.6, pp.141-148, Nov 2015.}

\subsection{\textbf{International Conference}}
\cvitem{1}{\textbf{Pranav S. Deshpande} and Kashif M.S., ``Enabling Touchless Control using Gesture Detection derived from Time of Flight (ToF) non-camera Sensor using INH algorithm", in \emph{4th Annual AMD Global Technical Authors Conference (GTAC '23) }, Dec. 2023}
\cvitem{2}{\textbf{Pranav S. Deshpande} and M. S. Manikandan, ``Improving Accuracy of Glottal Closure Instant Detection Methods in Nonstationary Noise", in \emph{Proc. IEEE Int. Conf. on Signal Processing and Integrated Networks (SPIN-2015)}, Feb. 2015}


\section{\textbf{Professional Awards }}
\cvitem{1} {AMD Innovation Expo 2024 Honorary Award}
\cvitem{2} {AMD Director's spotlight award for Q4 2024}
\cvitem{3}{AMD spotlight award for Q4 2023}
\cvitem{4}{Brigosha Best Performer of the year Award 2019}


%\section{\textbf{Masters Thesis}}

%\cvitem{Title}{\emph{Glottal Instant Detection from Speech \&  EGG Signals}}
%\cvitem{Supervisor}{Dr. M. Sabarimalai Manikandan (Asst. Prof.), SES, IIT Bhubaneswar}
%\cvitem{Description}{In this thesis, we attempt to develop an unified framework using variational mode decomposition (VMD) and autocorrelation feature based post-processing techniques for automatically detecting glottal closure instants (GCIs) and glottal opening instants (GOIs) from speech and EGG signals including both voiced speech and non-speech portions. The major objective of this thesis is to develop an unified VMD based filtering framework for extracting the glottal waveform feature signal meanwhile suppressing the background noises. In this work, we investigated a set of autocorrelation features for designing a post-processing technique to improve overall accuracy by reducing the number of false positives during non-speech portions without significantly reducing identification rate during voiced segment of EGG and speech signals.}

%----------------------------------------------------------------------------------------
%	COMMUNICATION SKILLS SECTION
%----------------------------------------------------------------------------------------

%\section{Communication Skills}
%
%\cvitem{2010}{Oral Presentation at the California Business Conference}
%\cvitem{2009}{Poster at the Annual Business Conference in Oregon}
%----------------------------------------------------------------------------------------
%	Extra SECTION
%----------------------------------------------------------------------------------------
\section{\textbf{Extras and Achievements}}

\subsection{\textbf{Academic}}
\cvitem{2013}{Secured \textbf{All India Rank 800} with score 649 in GATE 2013.}

\subsection{\textbf{Leadership and Management}}
\cvitem{2014-2015}{Worked as M.Tech. Electrical \textbf{Student's Representative at Career Development Cell} of IIT Bhubaneswar.}
\cvitem{2014-2015}{Worked as a \textbf{Mess Wing Counselor} at IIT Bhubaneswar Boys' Hostel; Madanpur.}
\cvitem{2008-2010}{Worked as \textbf{Field Officer, Assistant Coordinator and Coordinator} for the Publicity Comity of national level competition \textit{Brainwaves}.}
\cvitem{2008}{Worked as \textbf{Field Officer} for Orchestra Comity of cultural gathering \textit{Utopia}.}

\subsection{\textbf{Music and Literature}}
\cvitem{2021}{Won first prize at essay writing competition organized by Marathi Weekly Saptahik Vivek} %(https://www.evivek.com/Encyc/2021/1/12/Yes-only-Hindus-I-e-Indian.html).}
\cvitem{2017}{Authored an article on Marathi News Portal Smart Maharashtra} %(http://smartmaharashtra.online/aai/).}
\cvitem{2013}{Worked as \textbf{Assistant Music Director} for the Music Album \textit{Na Jaanu Kaisa Ishq Hai} .}%(https://www.youtube.com/watch?v=oBRXNt58pvE).}
\cvitem{2005}{Got \textbf{First Class} in Pune Bharat Gayan Samaj's first year classical Singing exam.}

%----------------------------------------------------------------------------------------
%	INTERESTS SECTION
%----------------------------------------------------------------------------------------

\section{\textbf{Interests}}

\renewcommand{\listitemsymbol}{-~} % Changes the symbol used for lists

\cvlistdoubleitem{Certified Yoga Teacher}{Playing Guitar and Harmonium}
\cvlistdoubleitem{Singing and composing songs, Poetry Writing}{Cooking, Swimming}
%----------------------------------------------------------------------------------------
%	LANGUAGES SECTION
%----------------------------------------------------------------------------------------

\section{\textbf{Languages}}

\cvitemwithcomment{Marathi}{Mother-tongue}{Conversationally fluent}
\cvitemwithcomment{English}{Fluent}{Medium of education after $10^{th}$ std. class}

%----------------------------------------------------------------------------------------
%	References SECTION
%----------------------------------------------------------------------------------------

%\section{\textbf{References}}
%\subsection{\textbf{1. Dr. M. Sabarimalai Manikandan}}
%\cvitem{Designation}{Assistant Professor,  School of Electrical Sciences, IIT Bhubaneswar}
%\cvitem{Email}{msm@iitbbs.ac.in}
%\cvitem{Address}{School of Electrical Sciences, IIT Bhubaneswar, Jatni-Kurdha Road, IIT Bhubaneswar, Odisha, India -752050}
%\subsection{\textbf{2. Dr. Barathram. Ramkumar}}
%\cvitem{Designation}{Assistant Professor,  School of Electrical Sciences, IIT Bhubaneswar}
%\cvitem{Email}{barathram@iitbbs.ac.in}
%\cvitem{Address}{School of Electrical Sciences, IIT Bhubaneswar, Jatni-Kurdha Road, IIT Bhubaneswar, Odisha, India -752050}

%----------------------------------------------------------------------------------------
%	COVER LETTER
%----------------------------------------------------------------------------------------

% To remove the cover letter, comment out this entire block

%\clearpage
%
%\recipient{HR Department}{Corporation\\123 Pleasant Lane\\12345 City, State} % Letter recipient
%\date{\today} % Letter date
%\opening{Dear Sir or Madam,} % Opening greeting
%\closing{Sincerely yours,} % Closing phrase
%\enclosure[Attached]{curriculum vit\ae{}} % List of enclosed documents
%
%\makelettertitle % Print letter title
%
%\lipsum[1-3] % Dummy text
%
%\makeletterclosing % Print letter signature

%----------------------------------------------------------------------------------------

\end{document}
